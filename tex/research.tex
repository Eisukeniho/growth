\documentclass[twocolumn]{jarticle}

\begin{document}

\title{プログラムの盗用検出}
\author{京都産業大学 コンピュータ理工学部 コンピュータサイエンス学科 仁保叡宥}
\date{令和2年10月1日}

\maketitle

%概要
\begin{abstract}
ここに'概要'を書く
\end{abstract}


%背景
\section*{背景}
プログラミング技術を向上させるためには,たくさんのプログラムを作成することが必要不可欠である.実際に大学のプログラミング演習の授業では,毎週プログラミング課題が出されている.学生はそれらの課題をこなすことでプログラミング技術を向上させることができる.ほとんどの学生がそれらの課題をこなす一方で,他人の作成したプログラムをコピペし,そのまま提出する学生も少なからず存在する.しかし,課題点を得るためだけならばコピペは有用なものであるが,コピペをすることでその学生のプログラミング技術は少しも向上しない.したがって,学生にとって他人のプログラムをコピペし提出することは有意義ではない.また,他の学生に自身が作成したプログラムをコピペさせる行為も,学生のプログラミング技術の向上を妨げる行為であるため,同様に有意義ではない.教員は,提出されたプログラミング課題に対して盗用の疑いを持ったとき,その疑いを確信に変えることはできない.そのため実際には,作成者である学生に対して,本当にその学生が作成したプログラムであるか問うことしかできない.また,教員は学生に対して本当に自身が作成したプログラムであるか問うとき,他の学生のプログラムと比較し,どのくらい類似しているかをポイントとして学生に問う.しかし実際には,他の学生のプログラムとどのくらい類似しているかというのは,根拠としては弱く,他人のプログラムを盗用した学生に対して,完全に問い詰めることは難しい.本研究では,類似性以外に,学生が他の学生のプログラムを盗用しているかどうかを判断するための材料の提案を目的としている.プログラミング演習課題をこなすことで,学生の作成したプログラムは,教科書プログラムとの類似度が低くなりそれぞれの学生の特有の書き方になっていくという仮説を立てた.それぞれの学生のプログラミング技術の成長度から,他の学生が作成したプログラムを盗用をしているかを判定する.また,類似したプログラムが複数存在する場合にも,それぞれの学生の成長度から,どちらが盗用されたプログラムでどちらが盗用したプログラムであるかの判定も行う.

%提案手法
\section*{提案手法}


\end{document}